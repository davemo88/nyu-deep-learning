\documentclass{article}

\usepackage{amsmath}
\usepackage{amssymb}
\usepackage{amsfonts}
\setlength{\parindent}{0in}
\begin{document}
\newcommand{\p}{\partial}
\newcommand{\xo}{x_{\text{out}}}
\newcommand{\Xo}{X_{\text{out}}}
\renewcommand{\xi}{x_{\text{in}}}
\renewcommand{\Xi}{X_{\text{in}}}
{\bf{Team Hodge Theaters}}\\
David Kasofsky  -- dk2353@nyu.edu\\
Jonay Tr\'enous -- jgt275@nyu.edu\\
\vspace*{2cm}

\centerline{\Huge{\bf{Assignment 1}}}
\section{Backpropagation}
\addtolength{\jot}{1em}
\subsection{}
By the Chain Rule, we can write
\begin{align*}
  \dfrac{\p E}{\p x_{in}} &= \dfrac{\p E}{\p x_{out}}\cdot \dfrac{\p x_{out}}{\p x_{in}} \\
 &= \dfrac{\p E}{\p x_{out}}\cdot \dfrac{\p(1+\exp(-x_{in}))^{-1}}{\p x_{in}}\\
&= \dfrac{\p E}{\p x_{out}}\cdot (1+\exp(-x_{in}))^{-2} \exp(-x_{in}) \\
&= \dfrac{\p E}{\p  x_{out}}\cdot \exp(-x_{in}) \cdot x_{out}^2 \\
\end{align*}
\subsection{}
First, consider the case $i\neq j$:
\begin{align*}
  \dfrac{\p(\Xo)_i}{\p (\Xi)_j} &\overset{\text{Prod. Rule}}{=} \dfrac{\p \exp(-\beta (\Xi)_i)}{\p (\Xi)_j}\left(\sum_k \exp(-\beta(\Xi)_k)\right)^{-1}+ \dfrac{\p \left(\sum_k \exp(-\beta(\Xi)_k)\right)^{-1}}{\p (\Xi)_j} \exp(-\beta (\Xi)_i)  \\
&= 0 + \beta \dfrac{\exp(-\beta (\Xi)_j) \exp(-\beta (\Xi)_i)}{\left(\sum_k \exp(-\beta(\Xi)_k)\right)^{2}} \\
&= \beta (\Xo)_j (\Xo)_i   
\end{align*}
In the opposite case $i=j$, the first term will not be zero:
\begin{align*}
  \dfrac{\p(\Xo)_i}{\p (\Xi)_i} &\overset{\text{Prod. Rule}}{=} \dfrac{\p \exp(-\beta (\Xi)_i)}{\p (\Xi)_i}\left(\sum_k \exp(-\beta(\Xi)_k)\right)^{-1}+ \dfrac{\p \left(\sum_k \exp(-\beta(\Xi)_k)\right)^{-1}}{\p (\Xi)_i} \exp(-\beta (\Xi)_i)  \\
&= \beta \dfrac{\exp(-\beta (\Xi)_i) \exp(-\beta (\Xi)_i)}{\left(\sum_k \exp(-\beta(\Xi)_k)\right)^{2}} - \beta \dfrac{\exp(-\beta (\Xi)_i)}{\sum_k \exp(-\beta(\Xi)_k)}\\
&= \beta \left((\Xo)^2_i - (\Xo)_i\right)   
\end{align*}

\end{document}

%%% Local Variables:
%%% mode: latex
%%% TeX-master: t
%%% End:
